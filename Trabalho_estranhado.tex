\documentclass[12pt]{article}
\usepackage{times}
\setlength\parindent{24pt}
\usepackage[margin=2cm]{geometry}
\usepackage[utf8]{inputenc}
\usepackage[portuguese]{babel}
\renewcommand{\baselinestretch}{1.5}

\begin{document}

\noindent
\large\textbf{HZ258 | Exercício de leitura \hfill Rafael Gonçalves (186062)}
\break\hfill

No parágrafo, Marx coloca que \em sociedade \em e \em indivíduo \em não existem exclusivamente.
O indivíduo em Marx é ser social, ser pensante, ser que vive em sociedade - incorporando e  objetivando novos elementos do e para o gênero humano.
A subjetividade do indivíduo é composta não só pelo indivíduo de forma separada da sociedade, mas sim através da incorporação de elementos do gênero humano.





\end{document}

