\documentclass[12pt]{article}
\usepackage{times}
\setlength\parindent{24pt}
\usepackage[margin=2cm]{geometry}
\usepackage[utf8]{inputenc}
\usepackage[portuguese]{babel}
\renewcommand{\baselinestretch}{1.5}

\begin{document}

\noindent
\large\textbf{HZ258 | Exercício de leitura \hfill Rafael Gonçalves (186062)}
\break\hfill

No parágrafo Marx coloca que \em indivíduo \em e \em sociedade \em não existem exclusivamente.
A sociedade não pode ser pensada como entidade separada ou independente dos indivíduos que a compõe.

O ser humano como elemento individual da sociedade é, em Marx, \em ser social \em.
Ele se assemelha ao \em animal \em (ser orgânico) na medida em que compartilha de suas carências - come, bebe, dorme, etc. -, trabalha e se reproduz.
Entretanto se diferencia dele porque não se esgota nesse nível de ser.
É ser que pensa, conceitua, se organiza em sociedade provendo para e absorvendo dela.
O animal \em trabalha \em para suprir suas carências imediatas individuais; enquanto que o ser humano pode, enquanto ser social, produzir para quaisquer necessidades outras.

No \em trabalho \em o ser humano externaliza sua subjetividade em objeto.
Objeto que - ao ser incorporado pela sociedade - suprassume o indivíduo, mas também todo gênero humano.

Marx chama de \em estranhamento \em o processo pelo qual essa relação entre ser humano e natureza (trabalho) cessa e passa a ser mediado pelo capital.
No \em trabalho estranhado \em a objetivação é feita de forma alheia ao indivíduo.
Esse alheiamento se dá em 3 formas.

1) Alheia o trabalhador do objeto produzido, na medida em que ele produz algo que não é para si, mas para um outro (estranho), que é o empregador.

2) Alheia o trabalhador do próprio trabalho, pois todo o processo produtivo acontece em situações alheias ao trabalhador.
O trabalhador sai de sua casa para trabalhar, usa ferramentas que não são as suas, segue as regras de um outro, etc.

3) Alheia o trabalhador do próprio gênero huimano, na medida que separa o trabalhador como classe, das próprias objetivações suas. O trabalhador não mais cria para o gênero humano, mas sim para um outro que limitará o acesso ao produto por intermédio do capital.


--- 


No parágrafo, Marx coloca que \em sociedade \em e \em indivíduo \em não existem exclusivamente.
O indivíduo em Marx é ser social, ser pensante, ser que vive em sociedade - incorporando e  objetivando novos elementos do e para o gênero humano.
A subjetividade do indivíduo é composta não só pelo indivíduo de forma separada da sociedade, mas sim através da incorporação de elementos do gênero humano.

\em Sociedade \em não é algo separado do \em indivíduo \em.
O indivíduo é \em ser social \em, ser que se reproduz e que pensa, ser que se organiza em sociedade.
Por isso a vida do ser humano é sempre externação e apropriação de elementos para e da sociedade.
A vida do indivíduo se manifesta necessariamente na sociedade, independentemente de 

---




\end{document}

