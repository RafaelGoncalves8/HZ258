\documentclass[12pt]{article}
\usepackage{times}
\setlength\parindent{24pt}
\usepackage[margin=2cm]{geometry}
\usepackage[utf8]{inputenc}
\usepackage[portuguese]{babel}
\renewcommand{\baselinestretch}{1.5}

\begin{document}

\noindent
\large\textbf{HZ258 | Exercício de leitura \hfill Rafael Gonçalves (186062)}
\break\hfill

No parágrafo, Marx coloca que \em indivíduo \em e \em sociedade \em não existem exclusivamente.
São unidade.
A sociedade não pode ser pensada como entidade separada ou independente dos indivíduos que a compõe e vice e versa.

Primeiramente, para Marx, o ser humano, visto como elemento individual da sociedade, é \em ser social\em.

O ser social, assim como os animais, dependem da \em natureza inorgânica \em para sua existência.
% Natureza e ser inorgânico são a mesma coisa em Marx?
A natureza é a base material para a constituição dos animais e, mais do que isso, é elemento que determina a existência destes através de sua relação de causalidade.
% A natureza é teleológica em Marx?
Mas apesar dos animais dependerem dela, a natureza não é suficiente para caracterizá-los.
Podemos dizer que a natureza constitúi o corpo inorgânico do ser social (e dos animais), mas não que ele se esgote nessa categoria.

O ser social é também \em animal \em (ser orgânico), na medida em que, diferente do ser inorgânico, compartilha de suas carências - come, bebe, dorme, etc. -, se reproduz e \em trabalha\em.
Mas a categoria do ser orgânico também não é suficiente para explicar o ser humano.
Enquanto que a consciência encontra um papel marginal no animal, no ser social seu papel é central.
A relação entre ser e natureza se da, no ser orgânico, de forma espontânea.
A atividade animal é puro reflexo da causalidade da natureza.

Por outro lado o ser social é um nível de ser que engloba as duas categorias anteriores, mas não se identifica com elas.
Essa categoria tem a consciência como elemento central na sua atividade.
O ser social é o ser que pensa, conceitua, abstrai.
Diferente do animal que trabalha para suprir carências imediatas individuais, o ser social é capaz de produzir para suprir quaisquer necessidades outras.
Ele se organiza em sociedade provendo e absorvendo dela.
Por isso, no trabalho, todo o gênero humano se desenvolve num movimento contínuo de \em suprassunção \em do ser.

O trabalho do ser social, em termos universais, é autoatividade, objetivação. 
É o elemento ineliminável que media as relações do ser humano com a natureza objetiva.
No trabalho, cada indivíduo produz a si mesmo na medida em que suprassumi seu próprio ser ao externalizar e objetivar sua subjetividade.
Mas também determina o ser como gênero, pois objetiva para a sociedade e todo indivíduo da mesma é afetado e incorpora as objetivações de todos os outros seres humanos.
Neste sentido sua consciência subjetiva é produto tardio da atividade objetiva do gênero humano.
Cada ser humano é determinado pela e determina a exterioridade objetiva.

Assim, o indivíduo pode ser entendido como ser genérico determinado.
O indivíduo é ser genérico particularizado enquanto que a sociedade é ser particular generalizado.

Mas há também um sentido particular (histórico) do trabalho.
O que Marx chama de \em trabalho estranhado\em.
\em Estranhamento \em é o processo pelo qual essa relação entre ser humano e natureza (trabalho) se transforma e passa a ser mediada pelo capital.

No trabalho estranhado a objetivação é feita de forma alheia ao indivíduo.
Esse alheiamento se dá em 3 formas:

1) Alheia o trabalhador do objeto produzido, na medida em que ele produz algo que não é para si, mas para um outro (estranho), que é o empregador.

2) Alheia o trabalhador do próprio trabalho, pois todo o processo produtivo acontece em situações alheias ao trabalhador.
O trabalhador sai de sua casa para trabalhar, usa ferramentas que não são as suas, segue as regras de um outro, etc.

3) Alheia o trabalhador do próprio gênero humano, na medida que separa o trabalhador como classe, das próprias objetivações suas. O trabalhador não mais cria universalmente para o gênero humano, mas sim para suprir suas carências. Produz para um outro que limitará o acesso dos indivíduos ao produto por intermédio do capital.

Marx conclui então que a característica de estranhamento presente no trabalho na sociedade capitalista é justamente causada pela existência da \em propriedade privada\em.
O trabalho estranhado seria a expressão subjetiva da propriedade privada.
E a própria existência da propriedade privada seria o que alheia o produto, o processo do trabalho e o próprio trabalhador do trabalhador como gênero (ou o ser do ser genérico).

Por fim, coloca-se que a superação da propriedade privada seria o \em comunismo\em.
Que seria a suprassunção da propriedade privada em propriedade privada universal, e portanto de trabalho estranhado em trabalho.
Assim, perde-se o caráter individual do trabalho estranhado e enfatiza-se o caráter social do trabalho e do ser-humano. Marx coloca ainda que o desenlvimento da sociedade capitalista para o comunismo seria um desenvolvimento necessário, mas não necessariamente o estágio final da sociedade humana.


\end{document}

