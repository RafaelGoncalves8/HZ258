\documentclass[12pt]{article}
\usepackage{times}
\setlength\parindent{24pt}
\usepackage[margin=2cm]{geometry}
\usepackage[utf8]{inputenc}
\usepackage[portuguese]{babel}
\renewcommand{\baselinestretch}{1.5}

\begin{document}

\noindent
\large\textbf{HZ258 | Exercício de leitura \hfill Rafael Gonçalves (186062)}
\break\hfill

No parágrafo, Marx coloca que \em indivíduo \em e \em sociedade \em não existem exclusivamente.
São unidade.
A sociedade não pode ser pensada como entidade separada ou independente dos indivíduos que a compõe e vice e versa.

Primeiramente, para Marx, o ser humano, visto como elemento individual da sociedade, é \em ser social\em.

O ser social, assim como os animais, dependem da \em natureza inorgânica \em (ser inorgânico) para sua existência.
% Natureza e ser inorgânico são a mesma coisa em Marx?
A natureza inorgânica é a base material para a constituição dos animais e mais que isso estes são determinados pela causalidade e contingência daquele.
% A natureza é teleológica em Marx?
Podemos dizer que a natureza constitúi o corpo inorgânico do ser social (e dos animais), mas não que ele se esgota nessa categoria.

O ser social é também \em animal \em (ser orgânico), na medida em que, diferente do ser inorgânico, compartilha de suas carências - come, bebe, dorme, etc. -, se reproduz e \em trabalha\em.
Mas a categoria do ser orgânico ainda não é suficiente para explicar o ser humano.
Enquanto que a consciência encontra no animal um papel marginal, no ser social seu papel é central.
A relação entre ser e natureza se da, no ser orgânico, de forma espontânea.
A atividade animal é puro reflexo da causalidade da natureza.

Por outro lado o ser social é um nível de ser que engloba as duas categorias anteriores, mas não se identifica com elas.
Essa categoria tem a consciência como elemento central na sua atividade.
O ser social é o ser que pensa, conceitua, abstrai.
Diferente do animal que trabalha para suprir carências imediatas individuais, o ser social é capaz de produzir para suprir quaisquer necessidades outras.
Ele se organiza em sociedade provendo e absorvendo dela.
Por isso, no trabalho, todo o gênero humano se desenvolve num movimento contínuo de \em suprassunção \em do ser.

O trabalho do ser social, em termos universais, é autoatividade, objetivação. 
É o elemento ineliminável que media as relações do ser humano com a natureza objetiva.
No trabalho, cada indivíduo produz a si mesmo a medida que suprassumi seu próprio ser ao externalizar e objetivar sua subjetividade.
Neste sentido sua consciência subjetiva é produto tardio de sua atividade objetiva.
Cada ser humano é determinado pela e determina a exterioridade objetiva.
Assim, o indivíduo passa a ser apenas ser genérico determinado.
O indivíduo morre, mas o gênero humano não.




---


O ser social é \em ser inorgânico \em porque é constituído e depende de toda natureza inorgânica para existir.
Sem a natureza inorgânica não há animal e não há ser humano.
Neste sentido toda a natureza constitui o corpo inorgânico do ser social.
Mas o ser social não é apenas natureza inorgânica.
Ele é também, além disso, \em animal \em (ser orgânico) na medida em que, diferente do ser inorgânico, compartilha de suas carências - come, bebe, dorme, etc. -, se reproduz e \em trabalha\em.
Entretando o ser humano visto como ser social também não se esgota na categoria de ser orgânico.

Trabalho em Marx significa objetivação, autoatividade humana, e é, portanto, o ineliminável elemento mediador das relações entre ser humano e natureza.
Porém

O ser humano também é ser que pensa, conceitua, se organiza em sociedade provendo para e absorvendo dela.
O animal trabalha para suprir suas carências imediatas individuais; enquanto que o ser humano pode, enquanto ser social, produzir para quaisquer necessidades outras.

No \em trabalho \em o ser humano externaliza sua subjetividade em objeto.
Objeto que - ao ser incorporado pela sociedade - suprassume o indivíduo, mas também todo gênero humano.

Marx chama de \em estranhamento \em o processo pelo qual essa relação entre ser humano e natureza (trabalho) cessa e passa a ser mediado pelo capital.
No \em trabalho estranhado \em a objetivação é feita de forma alheia ao indivíduo.
Esse alheiamento se dá em 3 formas.

1) Alheia o trabalhador do objeto produzido, na medida em que ele produz algo que não é para si, mas para um outro (estranho), que é o empregador.

2) Alheia o trabalhador do próprio trabalho, pois todo o processo produtivo acontece em situações alheias ao trabalhador.
O trabalhador sai de sua casa para trabalhar, usa ferramentas que não são as suas, segue as regras de um outro, etc.

3) Alheia o trabalhador do próprio gênero huimano, na medida que separa o trabalhador como classe, das próprias objetivações suas. O trabalhador não mais cria para o gênero humano, mas sim para um outro que limitará o acesso ao produto por intermédio do capital.


--- 


No parágrafo, Marx coloca que \em sociedade \em e \em indivíduo \em não existem exclusivamente.
O indivíduo em Marx é ser social, ser pensante, ser que vive em sociedade - incorporando e  objetivando novos elementos do e para o gênero humano.
A subjetividade do indivíduo é composta não só pelo indivíduo de forma separada da sociedade, mas sim através da incorporação de elementos do gênero humano.

\em Sociedade \em não é algo separado do \em indivíduo \em.
O indivíduo é \em ser social \em, ser que se reproduz e que pensa, ser que se organiza em sociedade.
Por isso a vida do ser humano é sempre externação e apropriação de elementos para e da sociedade.
A vida do indivíduo se manifesta necessariamente na sociedade, independentemente de 

O indivíduo é ser genérico determinado. O indivíduo particular morre. O gênero humano universal não.
---




\end{document}

