\documentclass[12pt]{article}
\usepackage{times}
\setlength\parindent{24pt}
\usepackage[margin=2cm]{geometry}
\usepackage[utf8]{inputenc}
\usepackage[portuguese]{babel}
\renewcommand{\baselinestretch}{1.5}

\begin{document}

\noindent
\large\textbf{HZ258 | Exercício de leitura \hfill Rafael Gonçalves (186062)}
\break\hfill

Propriedade privada e divisão social do trabalho como expressões da mesma coisa (o primeiro no que diz respeito ao produto do trabalho e o segundo no que diz respeito das relações humanas).

No desenvolvimento atual da sociedade há uma separação tão grande entre autoatividade e produção de vida material que a vida material (reprodução da vida?) aparece como finalidade a criação de vida material, o trabalho, expressão negativa da autoatividade, aparece como meio.

Portanto a apropriação da totalidade das forças produtivas (por extensão dos meios de produção) se torna tarefa necessária não para se chegar na autoatividade, mas primeiro para assegurar sua própria existência (vida?)

Somente os proletários atuais, excluídos de autoatividade, estão em condições de impor a autoatividade plena.

Outras apropriações revolucionárias anteriores fora insuficientes pois a apropriação de um instrumento de produção limitado gerava sempre uma nova limitação.

Seu instrumento de produção tornava-se propriedade, mas ele mesmo permanecia subsumido à divisão de trabalho (propriedade privada).

Nas apropriações anteriores uma massa de indivíduos permaneciam subsumidos a um único instrumento de produção, na apropriação pelos proletários, uma massa de instrumentos de produção tem de ser subsumida a cada indivíduo. A propriedade é subsumida a todos.

O intercâmbio universal não pode ser subsumido aos indivíduos senão na condição de ser subsumido a todos.

Infraestrutura material é base da superestrutural ideal.

A história é a história do movimento real (material) dos indivíduos que a compões (forças produtivas e forma de intercâmbio). Não existe uma história do direito, da filosofia, enfim, de ideias (representações?).

\end{document}

