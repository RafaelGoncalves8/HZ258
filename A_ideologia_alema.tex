\documentclass[12pt]{article}
\usepackage{times}
\setlength\parindent{24pt}
\usepackage[margin=2cm]{geometry}
\usepackage[utf8]{inputenc}
\usepackage[portuguese]{babel}
\renewcommand{\baselinestretch}{1.5}

\begin{document}

\noindent
\large\textbf{HZ258 | Exercício de leitura II \hfill Rafael Gonçalves (186062)}
\break\hfill

N'\em A ideologia alemã\em, Marx e Engels visam criticar os assim chamados jovens hegelianos (B. Bauer, M. Stirner e L. Feuerbach) mostrando que suas filosofias - tidas como revolucionárias na época - são falhas, pois não quebram completamente com a noção hegeliana de que a história seria uma história das ideias (ou do espírito).
Para tal, os autores fazem uma exposição do que seria o movimento de gênese e desenvolvimento da história, mostrando que o que move a história desde o princípio é a \em atividade humana \em no plano material.
O desenvolvimento da religião, do direito, da filosofia, etc. seriam apenas consequência do desenvolvimento da matéria - ou ainda das formas de produção e relações humanas. Em suma, a atividade humana (e não o espírito) é o sujeito da história.

O ponto atual de desenvolvimento da história seria então o período posterior à burguesia assumir o papel de classe dominante, em que , há o aparecimento de uma economia global, etc. % TODO
Esse momento em que a sociedade se encontra se expressa no objeto produzido pela atividade humana como \em propriedade privada \em e nas relações sociais como \em divisão social do trabalho \em.

No desenvolvimento atual da sociedade, há uma separação tão grande entre autoatividade e produção de vida material (trabalho) que a vida material (ou ainda condições mínimas de existência do trabalhador) aparece como finalidade e a criação de vida material propriamente dita aparece como meio.
Trabalho como trabalho estranhado, ou expressão negativa da autoatividade.

Com isto, 

--- 

Propriedade privada e divisão social do trabalho como expressões da mesma coisa (o primeiro no que diz respeito ao produto do trabalho e o segundo no que diz respeito das relações humanas).

No desenvolvimento atual da sociedade há uma separação tão grande entre autoatividade e produção de vida material que a vida material (reprodução da vida?) aparece como finalidade a criação de vida material, o trabalho, expressão negativa da autoatividade, aparece como meio.

Portanto a apropriação da totalidade das forças produtivas (por extensão dos meios de produção) se torna tarefa necessária não para se chegar na autoatividade, mas primeiro para assegurar sua própria existência (vida?)

Somente os proletários atuais, excluídos de autoatividade, estão em condições de impor a autoatividade plena.

Outras apropriações revolucionárias anteriores fora insuficientes pois a apropriação de um instrumento de produção limitado gerava sempre uma nova limitação.

Seu instrumento de produção tornava-se propriedade, mas ele mesmo permanecia subsumido à divisão de trabalho (propriedade privada).

Nas apropriações anteriores uma massa de indivíduos permaneciam subsumidos a um único instrumento de produção, na apropriação pelos proletários, uma massa de instrumentos de produção tem de ser subsumida a cada indivíduo. A propriedade é subsumida a todos.

O intercâmbio universal não pode ser subsumido aos indivíduos senão na condição de ser subsumido a todos.

Infraestrutura material é base da superestrutural ideal.

A história é a história do movimento real (material) dos indivíduos que a compões (forças produtivas e forma de intercâmbio). Não existe uma história do direito, da filosofia, enfim, de ideias (representações?).

\end{document}

