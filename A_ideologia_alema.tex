\documentclass[12pt]{article}
\usepackage{times}
\setlength\parindent{24pt}
\usepackage[margin=2cm]{geometry}
\usepackage[utf8]{inputenc}
\usepackage[portuguese]{babel}
\renewcommand{\baselinestretch}{1.5}

\begin{document}

\noindent
\large\textbf{HZ258 | Exercício de leitura II \hfill Rafael Gonçalves (186062)}
\break\hfill

N'\em A ideologia alemã\em, Marx e Engels visam criticar os assim chamados jovens hegelianos (B. Bauer, M. Stirner e L. Feuerbach) mostrando que suas filosofias - tidas como revolucionárias na época - são falhas, pois não quebram completamente com a noção hegeliana de que a história seria uma história das ideias (ou do espírito).
Para tal, os autores fazem uma exposição do que seria o movimento de gênese e desenvolvimento da história, mostrando que o que move a história desde o princípio é a \em atividade humana \em no plano material.
O desenvolvimento da religião, do direito, da filosofia, etc. seriam apenas consequências do desenvolvimento da matéria - ou ainda das formas de produção e relações humanas. Em suma, a atividade humana (e não o espírito) é o sujeito da história.

O ponto atual de desenvolvimento da história seria então a sociedade capitalista burguesa, em que a burguesia se estabelece como classe dominante, há uma transição da predominância de capital fixo (renda de terra) para capital móvel (mercadorias), há o aparecimento de uma economia global (intercâmbio universal), etc.
Esse momento em que a sociedade se encontra se caracteriza no que diz respeito à forma de atividade humana como \em divisão social do trabalho \em e no que diz respeito ao produto dessa atividade como \em propriedade privada\em .

A atividade do sujeito seria o que media sua interação com o mundo externo. Através dela o indivíduo tem o potencial de transformar o mundo material e igualmente ser transformado por este. Neste sentido o ser humano (ou o \em gênero humano\em) seria o produtor de si mesmo na medida que ele modifica a realidade material que o determina através desse movimento de autoatividade (ou objetivação, trabalho no sentido universal).

Mas no contexto atual, da divisão social do trabalho, há uma separação tão grande entre autoatividade e produção de vida material (trabalho, no seu sentido histórico, estranhado) que a vida material (condições mínimas de existência do trabalhador) aparece como finalidade de sua atividade e a criação de vida material (objetivação) propriamente dita aparece apenas como meio.
O proletário expressa sua atividade na forma de trabalho estranhado - expressão negativa da autoatividade -, trabalho esse que tem por fim suprir as suas carências vitais.

Portanto, num contexto em que o trabalho se mostra como único vínculo do trabalhador com sua própria existência, há a necessidade da apropriação do que condiciona essa a atividade humana primeiramente para que os trabalhadores assegurem sua sobrevivência.

Essa mudança só se dará com a apropriação do que condiciona essa forma de trabalho universalmente, com a apropriação da totalidade das forças produtivas (e portanto dos instrumentos materiais de produção).
Essa apropriação só é possível agora, porque os trabalhadores estão completamente excluídos de autoatividade, sendo assim possível a aquisição da autoatividade plena e universal na apropriação dos meios de produção.

Outras tentativas de apropriação falharam, pois nunca houve uma apropriação da totalidade das forças produtivas, somente de uma parcela delas.
As forças produtivas nos instrumentos de produção tomados eram parciais e o intercâmbio era também limitado, fazendo com que a apropriação da capacidade produtiva fosse limitada e não universal. Um instrumento de produção tornou-se sua posse, mas não a totalidade deles, ou seja, o indivíduo ainda está subsumido à propriedade privada e à divisão social do trabalho.

O texto mostra que agora existem condições materiais para a apropriação total dessas forças produtivas. A suprassunção da propriedade privada só se dará na forma de propriedade privada universal - submissão de toda propriedade a cada indivíduo - da mesma forma que modo de intercâmbio universal só pode ser suprassumido se é submetido a totalidade de indivíduos. Esse seria o movimento de suprassunção do modo atual de produção.

\end{document}

