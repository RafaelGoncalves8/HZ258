\documentclass[12pt]{article}
\usepackage{times}
\setlength\parindent{24pt}
\usepackage[margin=2cm]{geometry}
\usepackage[utf8]{inputenc}
\usepackage[portuguese]{babel}
\renewcommand{\baselinestretch}{1.5}

\begin{document}

\noindent
\large\textbf{HZ258 | Exercício de leitura II \hfill Rafael Gonçalves (186062)}
\break\hfill

N'\em A ideologia alemã\em, Marx e Engels visam criticar os assim chamados jovens hegelianos (B. Bauer, M. Stirner e L. Feuerbach) mostrando que suas filosofias - tidas como revolucionárias na época - são falhas, pois não quebram completamente com a noção hegeliana de que a história seria uma história das ideias (ou do espírito).
Para tal, os autores fazem uma exposição do que seria o movimento de gênese e desenvolvimento da história, mostrando que o que move a história desde o princípio é a \em atividade humana \em no plano material.
O desenvolvimento da religião, do direito, da filosofia, etc. seriam apenas consequências do desenvolvimento da matéria - ou ainda das \em forças produtivas \em e das relações humanas.
Em suma, a atividade humana (e não o espírito) é o sujeito da história.

O ponto atual de desenvolvimento da história seria então a sociedade capitalista burguesa, em que a burguesia se estabelece como detentora dos meios de produção, há uma transição da predominância de capital fixo (renda de terra) para capital móvel (mercadorias), há o aparecimento de uma economia global (intercâmbio universal), etc.
Esse momento em que a sociedade se encontra se caracteriza, no que diz respeito às relações humanas, como \em divisão social do trabalho \em e, no que diz respeito ao produto de sua atividade, como \em propriedade privada\em .

A atividade do ser humano media sua interação com o mundo externo. Através dela o indivíduo tem o potencial de transformar o mundo material e consequentemente ser transformado por este.
Neste sentido, o ser humano (ou o \em gênero humano\em) seria o produtor de si mesmo na medida que ele modifica a realidade material que o determina através dessa objetivação que ocorre na atividade (movimento que os autores chamam de \em autoatividade\em).

Mas no contexto atual, da divisão social do trabalho, há uma separação grande entre autoatividade e produção de vida material  - trabalho, no seu sentido histórico, estranhado. Assim, a vida material do próprio trabalhador aparece como finalidade de sua atividade e a criação de vida material propriamente dita aparece apenas como meio para isso.
O proletário expressa sua atividade na forma de trabalho estranhado - em contraposição à autoatividade -, trabalho esse que tem por fim suprir as suas carências vitais.

Portanto, num contexto em que o trabalho se mostra como único vínculo entre o trabalhador e sua própria existência, há a necessidade da apropriação daquilo que possibilita o exercício da atividade humana primeiramente para que os trabalhadores assegurem sua própria sobrevivência.
Esse condicionante seriam as forças produtivas: a \em força de trabalho \em de cada trabalhador e os intrumentos materiais de produção (\em meios de produção\em ).

Essa apropriação deve ser necessariamente da totalidade das forças produtivas (ou do que ainda falta delas aos trabalhadores: os meios de produção).
Neste sentido, essa apropriação só é possível agora, porque os trabalhadores estão completamente excluídos de autoatividade e portanto em condições de exercer a autoatividade plena.

A apropriação dos meios de produção significaria o desenvolvimento de uma totalidade da potência de autoatividade por parte dos trabalhadores.
Toda força produtiva reside nos meios de produção e na força de trabalho dos trabalhadores, então a apropriação destes significaria um desenvolvimento total das capacidades produtivas ou a autoatividade plena.

Outras tentativas de apropriação falharam, pois nunca houve uma apropriação da totalidade das forças produtivas, somente de uma parcela delas.
As forças produtivas nos instrumentos de produção tomados eram parciais e o intercâmbio era também limitado e portanto a capacidade de forças produtivas era limitada e não universal.
O trabalhador adquiriu uma parcela dos instrumentos de produção (que são então sua posse), mas continua submetido à propriedade privada e portanto à divisão social do trabalho.

A submissão de todos os meios de produção aos trabalhadores por outro lado significaria a elevação da propriedade privada em propriedade privada universal e a submissão do intercâmbio universal à totalidade de indivíduos, sendo assim a suprassunção efetiva do modo de produção atual.


\end{document}

